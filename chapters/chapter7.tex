\chapter{Application Results and User Interface}
\setlength{\parindent}{0pt}
\setlength{\parskip}{6pt}
{\setstretch{1.5}

This chapter presents the application-level results of the \textbf{ResQNow} emergency response system as observed through its user interface. Unlike the implementation chapter, which focused on internal logic, system architecture, and technical workflows, this chapter emphasizes the \textbf{end-user experience} by demonstrating how the application appears, behaves, and responds during real emergency scenarios.

The objective of this chapter is to validate the usability, clarity, and effectiveness of the system from the user’s perspective. It highlights how users interact with various screens, receive feedback during critical moments, and are guided through emergency situations with minimal effort. By focusing on visible outcomes and interaction flow, this chapter confirms that the system is intuitive, responsive, and suitable for real-world emergency deployment.

% -----------------------------------------------------------------------------
\section{Login and Signup Interface}

The login and signup interface represents the entry point to the ResQNow application and serves as the first interaction between the user and the system. It allows users to securely authenticate themselves before accessing any emergency-related services, ensuring controlled and personalized usage of the application. Users can either log in using previously registered credentials or create a new account by providing basic details such as full name, email address, and password.

Figure~\ref{fig:ui_login} illustrates the login and signup screen of the ResQNow application. The interface is designed with a clean, minimal, and distraction-free layout to ensure ease of access even under stressful emergency conditions. Clear input fields, large action buttons, and intuitive navigation enhance usability and reduce user effort. Upon successful authentication, the user is seamlessly redirected to the main dashboard, confirming system readiness and enabling immediate access to emergency response features.

\begin{figure}[H]
    \centering
    \includegraphics[width=0.83\textwidth]{chapters/ui_login.png}
    \caption{Login and signup interface of the ResQNow application}
    \label{fig:ui_login}
\end{figure}

% -----------------------------------------------------------------------------
\section{User Dashboard and Emergency Contact Management}

After successful authentication, the user is presented with the ResQNow dashboard, which serves as the central control interface for all emergency-related actions. The dashboard provides quick access to core features such as AI assistance, BLE connectivity, AR guidance, and SOS triggering. It also displays recent SOS alerts along with their priority levels for improved situational awareness during critical emergency situations.

Figure~\ref{fig:ui_dashboard} shows the dashboard interface of the ResQNow application. An important functionality available on this screen is the ability to add and manage \textbf{emergency contacts}. These contacts are automatically notified whenever an SOS is generated, ensuring that trusted individuals such as family members or close friends are immediately informed during emergencies and can respond without unnecessary delay.

\begin{figure}[H]
    \centering
    \includegraphics[width=0.78\textwidth]{chapters/ui_dashboard.png}
    \caption{ResQNow dashboard showing quick actions, recent alerts, and emergency contact management}
    \label{fig:ui_dashboard}
\end{figure}

% -----------------------------------------------------------------------------
\section{AI Chatbot Interaction and Automatic SOS Generation}

One of the most significant user-facing features of ResQNow is the AI-powered emergency chatbot. The chatbot initiates a supportive and calming conversation with the user to understand the emergency context. It asks relevant questions, provides reassurance, and continuously analyzes user responses to assess risk.

Figure~\ref{fig:ui_ai_chat} illustrates the AI chatbot interaction along with the corresponding SOS alert details generated by the system. Based on conversational analysis, the AI automatically identifies the type and severity of the emergency. In high-risk situations, such as suspected cardiac events, the system automatically triggers an SOS without requiring manual confirmation. The alert is visible to multiple responders, and once one responder accepts the alert, others are notified that help is already in progress. The interface also provides a \textbf{“View Location on Map”} option, which opens Google Maps with the live location of the affected user for accurate navigation.

\begin{figure}[H]
    \centering
    \includegraphics[width=.92\textwidth]{chapters/ui_ai_conversation.jpg}
    \caption{AI chatbot interaction and the corresponding SOS alert details generated by the system}
    \label{fig:ui_ai_chat}
\end{figure}

% -----------------------------------------------------------------------------
\section{Augmented Reality CPR Assistance Output}

When immediate first-aid intervention is required, ResQNow activates real-time AR-based CPR guidance to assist the user during cardiac emergencies. From the user’s perspective, visual markers, directional cues, and instructional overlays are superimposed directly onto the live camera feed, guiding correct hand placement, compression depth, and compression rhythm in real time under stressful emergency conditions.

Figure~\ref{fig:ui_ar_cpr} shows the AR-based CPR guidance interface. The interface provides clear step-by-step instructions and visual feedback that help users maintain proper technique and consistency throughout the procedure. By translating medical instructions into intuitive visual cues, the system minimizes cognitive load and enables even untrained bystanders to confidently perform CPR. This reduces hesitation, improves procedural accuracy, and increases the likelihood of effective intervention during critical, time-sensitive emergency situations with enhanced user confidence and situational awareness.

\begin{figure}[H]
    \centering
    \includegraphics[width=0.82\textwidth]{chapters/AndroidCPR (1).jpg}
    \caption{Augmented reality CPR guidance with visual hand-placement indicators}
    \label{fig:ui_ar_cpr}
\end{figure}

% -----------------------------------------------------------------------------
\section{Augmented Reality AED Assistance Output}

In addition to CPR guidance, ResQNow provides AR-based assistance for Automated External Defibrillator (AED) usage. The interface visually guides users through device activation, pad placement, and safety precautions in a step-by-step manner, ensuring clarity and confidence even for users with no prior medical training.

Figure~\ref{fig:ui_ar_aed} illustrates the AR-based AED guidance interface. The visual overlays significantly reduce hesitation and ensure correct AED usage during time-critical cardiac emergencies by presenting clear instructions, safety warnings, and procedural sequencing, thereby improving the chances of survival before professional medical help arrives.

\begin{figure}[H]
    \centering
    \includegraphics[width=0.75\textwidth]{chapters/AndroidAED (1).jpg}
    \caption{Augmented reality AED guidance displaying device operation and pad placement}
    \label{fig:ui_ar_aed}
\end{figure}

% -----------------------------------------------------------------------------
\section{Summary}

This chapter presented the application-level results of the ResQNow emergency response system through its user interface. The results demonstrate secure authentication, intuitive dashboard navigation, effective emergency contact integration, AI-driven conversational emergency detection with automatic SOS generation, live location sharing for responders, and immersive AR-based first-aid assistance for CPR and AED usage. By focusing exclusively on user-visible behavior and interaction flow, this chapter validates the system’s usability, clarity, and readiness for real-world emergency deployment.

}
