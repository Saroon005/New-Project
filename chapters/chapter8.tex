\chapter{Conclusion and Future Enhancements}
\setlength{\parindent}{0pt}
\setlength{\parskip}{6pt}
{\setstretch{1.5}

This chapter summarizes the project outcomes and evaluates its effectiveness as an intelligent emergency response system. It highlights major contributions and discusses future enhancements to extend scalability and real-world applicability.
\section{Conclusion}

The \textbf{ResQNow} project integrates \textbf{AI}, \textbf{AR}, and \textbf{mobile communication technologies} to improve emergency response effectiveness. It enables rapid SOS activation, automated severity classification, GPS tracking, and AR-guided first aid, allowing untrained bystanders to assist confidently. Testing confirms low-latency alertsúSOS delivery, reliable BLE-based offline communication, and a secure, scalable Flutter–Firebase–FastAPI architecture suitable for real-world deployment.
\section{Future Scope}

The following enhancements can be considered to further improve the system:
\begin{itemize}
    \item Deployment of optimized on-device AI models for fully offline emergency detection and classification
    \item Expansion of multi-hop BLE mesh networking with responder verification and reward mechanisms
    \item Multilingual and voice-driven AI assistant for hands-free emergency interaction and guidance
    \item Direct integration with hospitals, ambulance dispatch systems, wearable devices, and physiological sensors
    \item Advanced 3D AR-based first-aid guidance combined with predictive analytics and emergency risk mapping
\end{itemize}

}